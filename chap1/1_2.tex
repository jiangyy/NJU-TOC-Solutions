\problem{Prove: for any $k\in\mathbb{N}^+, f:\mathbb{N}^k\to\mathbb{N}$,
there always exists $h$ satisfying $f(\mathbf{x})<\|\mathbf{x}\|+h$ if $f\in\mathcal{BF}$.}
\begin{pf} \rm
We perform a structural induction on the constructive length $\ell$ of basic function $f$.

When $\ell=0$, $f\in\mathcal{IF}$. Thus $f(x) \leq S(x) < x + 2$ for all $x$.
Let $h_0 = 2$.

We assume when $0\leq \ell \leq n$, all functions $f$ with constructive length
no longer than $\ell$ satisfy $f(\mathbf{x})<\|\mathbf{x}\|+h_n$.

In the case of $\ell = n + 1$, assume that $f$ is constructed by sequence $f_0,f_1,\ldots,f_n,f$.
If $f\in\mathcal{IF}$, it is trivial that $f(x) \leq S(x) < \|\mathbf{x}\|+2h_{n}$.
Elsewise, $f = \textrm{Comp}^m_k[f_{i_0},f_{i_1},\ldots,f_{i_k}]$.
By inductive hypothesis we have $f_{i_j} < h_n$ for all $j$, thus
$f(\mathbf{x}) < \max\{f_{i_j}(\mathbf{x})\} + h_n < \|\mathbf{x}\| + 2h_n$.
Therefore, by letting $h=2^{\ell + 1}$,  $f(\mathbf{x})<\|\mathbf{x}\|+h$ always
holds. \qed

\end{pf}

