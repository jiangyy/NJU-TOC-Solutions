\documentclass[11pt]{article}
\usepackage{amssymb}
\usepackage{amsthm}
\usepackage{fontspec,xunicode,xltxtra}
\usepackage{titlesec}
\usepackage{indentfirst}
\usepackage[BoldFont]{xeCJK}
\usepackage{url}
\usepackage[superscript]{cite}
\usepackage[b5paper]{geometry}
\usepackage[vlined,boxed,linesnumbered]{algorithm2e}
\usepackage{fancyhdr}
\usepackage{graphicx}
\usepackage{listings}

% \XeTeXinputencoding "cp936"

\setCJKfamilyfont{kai}{KaiTi_GB2312}
\setCJKfamilyfont{hei}{STXihei}
\setCJKmainfont{SimSun}

\pagestyle{fancy}
\rhead{{\sf\thepage}}
\lhead{\sf ``Introduction to Models of Computation'' Solutions}

\renewcommand{\contentsname}{\hei 目录}
\renewcommand{\figurename}{\kai 图}
\renewcommand{\algorithmcfname}{\kai 算法}

\newcommand{\kai}{\CJKfamily{kai}}
\newcommand{\hei}{\CJKfamily{hei}}

\newcommand\TT{\rule{0pt}{2.6ex}}
\newcommand\BB{\rule[-1.2ex]{0pt}{0pt}}

\renewcommand\refname{\hei 参考文献}

%\setlength{\parindent}{2.2em}

\lstset {
        basicstyle = \sf,
        language = Java,
        tabsize = 4,
        columns = flexible
}

\usepackage{enumitem}
\setenumerate[1]{itemsep=3pt,partopsep=0pt,parsep=\parskip,topsep=5pt}
\setitemize[1]{itemsep=3pt,partopsep=0pt,parsep=\parskip,topsep=5pt}
\setdescription{itemsep=3pt,partopsep=0pt,parsep=\parskip,topsep=5pt}
\usepackage{sectsty}
\allsectionsfont{\hei}

\newcommand{\IF}{\mathcal{IF}}
\newcommand{\EF}{\mathcal{EF}}
\newcommand{\PRF}{\mathcal{PRF}}
\newcommand{\RF}{\mathcal{RF}}
\newcommand{\pg}{\mathrm{pg}}
\newcommand{\rs}{\mathrm{rs}}
\newcommand{\eq}{\mathrm{eq}}
\newcommand{\ep}{\mathrm{ep}}
\newcommand{\mod}{\textrm{ mod }}


\begin{document}

\newtheorem{thm}{定理}
\newtheorem{deff}[thm]{定义}
\newtheorem*{pf}{Proof}
\newtheorem{lem}[thm]{引理}

\title{\hei \bf ``Introduction to Models of Computation'' Solutions}
\author {Yanyan Jiang}
\date{\kai Spring, 2012}

\maketitle

\section{Chapter 1}

\subsection{Prove: for any fixed $k$, unary number theoretic function $x+k\in\mathcal{BF}$.}
\begin{pf} \rm We have $+_0 = P_1^1$ and
$+_k=\underbrace{S\circ S\circ \ldots \circ S}_{\textrm{$k-1$ times}}\in\mathcal{BF}$
for all $k\geq 1$. \qed
\end{pf}

\subsection{Prove: for any $k\in\mathbb{N}^+, f:\mathbb{N}^k\to\mathbb{N}$,
there always exists $h$ satisfying $f(\mathbf{x})<\|\mathbf{x}\|+h$ if $f\in\mathcal{BF}$.}
\begin{pf} \rm
We perform a structural induction on the constructive length $\ell$ of basic function $f$.

When $\ell=0$, $f\in\mathcal{IF}$. Thus $f(x) \leq S(x) < x + 2$ for all $x$.
Let $h_0 = 2$.

We assume when $0\leq \ell \leq n$, all functions $f$ with constructive length
no longer than $\ell$ satisfy $f(\mathbf{x})<\|\mathbf{x}\|+h_n$.

In the case of $\ell = n + 1$, assume that $f$ is constructed by sequence $f_0,f_1,\ldots,f_n,f$.
If $f\in\mathcal{IF}$, it is trivial that $f(x) \leq S(x) < \|\mathbf{x}\|+2h_{n}$.
Elsewise, $f = \textrm{Comp}^m_k[f_{i_0},f_{i_1},\ldots,f_{i_k}]$.
By inductive hypothesis we have $f_{i_j} < h_n$ for all $j$, thus
$f(\mathbf{x}) < \max\{f_{i_j}(\mathbf{x})\} + h_n < \|\mathbf{x}\| + 2h_n$.
Therefore, by letting $h=2^{\ell + 1}$,  $f(\mathbf{x})<\|\mathbf{x}\|+h$ always
holds. \qed

\end{pf}

\subsection{Prove: binary number theoretic function $x+y\notin\mathcal{BF}$.}
\begin{pf} \rm
 We have already proved that for any $k\in\mathbb{N}^+, f:\mathbb{N}^k\to\mathbb{N}$,
there always exists $h$ satisfying $f(\mathbf{x})<\|\mathbf{x}\|+h$ if $f\in\mathcal{BF}$.

If $x+y\in\mathcal{BF}$, there is $h$ such that
$x + x = 2x = 2\|\mathbf{x}\| < \|\mathbf{x}\| + h$,
which implies $x < h$, leading to contradiction. \qed

\end{pf}

\subsection{Prove: binary number theoretic function $x-y\notin\mathcal{BF}$.}
\begin{pf} \rm
Since $\mathrm{pred} = \mathrm{Comp}^1_2[P_1^1, S\circ Z]$, proving $\mathrm{pred}\notin
\mathcal{BF}$ is enough to show $x-y\notin\mathcal{BF}$.
Assume there exists shortest construction procedure
$f_0, f_1, \ldots, f_n, \mathrm{pred}$. There are two cases:

Case 1. $f_n\in\{S,Z,P\}$ is not the case.

Case 2. $f_n$ is a composition of $S,Z$ or $P$. $f_n$ cannot be composition of $S$ because
$S(x)>0$ for all $x$, and $\mathrm{pred}(1)=0$. Also, $f_n$ cannot be composition of 
$Z$ because $\mathrm{pred}(x)$
can be arbitrarily large. Finally, $f_n$ cannot be composition of $P$ because this contradicts
the shortest construction assumption. \qed

\end{pf}

\subsection{Let $\mathrm{pg}(x, y)=2^x(2y+1)-1$. Prove that there exists 
elementary function $K(x)$ and $L(x)$ such that $K(\mathrm{pg}(x, y)) = x,
L(\mathrm{pg}(x, y)) = y$ and $\mathrm{pg}(K(z),L(z))=z$.}

\begin{pf} \rm
 Let $\displaystyle K(x) = \mathrm{ep}_0 (x + 1), L(x) = 
\frac{1}{2} \left( \frac{x + 1}{2^{K(x)}} - 1 \right) $, we have
 
$\displaystyle \mathrm{pg}(K(z),L(z)) = 
2^{\mathrm{ep}_0(z+1)} \left( \frac{z + 1}{2^{\mathrm{ep}_0(z+1)}} \right) - 1
=z
$. \qed


\end{pf}

\subsection{Let $f:\mathbb{N}\to\mathbb{N}$. Prove that $f$ could be left function in
a pairing function if and only if $|\{x\in\mathbb{N} : f(x) = i\}|=\aleph_0$
for all $i\in\mathbb{N}$.}

\begin{pf} \rm
 The necessity is trivial by a simple contradiction.
 For the sufficiency, $|\{x\in\mathbb{N} : f(x) = i\}|=\aleph_0$ implies that
 there exists onto mapping
 $f_i : N_i\to \mathbb{N}$ such that $N_i = \{x~ | ~f(x) = i\}$ for all $i$,
 which implies that $f_i^{-1}$ exists for all $i$.
 By letting $\mathrm{pg}(x, y) = f_x^{-1}(y)$,
 we have $K(z) = f(f_x^{-1}(z)) = x$ and $L(z) 
 = f_x(z) = f_x( f_x^{-1}(y) ) = y$. \qed

\end{pf}

\subsection{Prove that all elementary function can be generated by applying
composition and $\prod_{i=n}^m [\cdot]$ operator.}

\begin{pf} \rm
  We first build some function by the conditioning ability of $\Pi$:

\[
 \begin{array}{c}
  \displaystyle N(x) = \prod_{i=1}^{x} Z(i), N^2(x) = \prod_{i=1}^{N(x)} Z(i) \\
  \displaystyle \mathrm{leq}(x, y) = \prod_{i = x}^y Z(i), 
  \displaystyle \mathrm{geq}(x, y) = \prod_{i = y}^x Z(i) \\
  \displaystyle \mathrm{gt}(x, y) = N(\mathrm{leq}(x, y)), 
  \displaystyle \mathrm{lt}(x, y) = N(\mathrm{geq}(x, y)). \\
 \end{array}
\]

  Then, we can conjunct and disjunct between predicates by

\[
  \mathrm{\land}(x, y) = \prod_{i = 1}^{N(x)} y, 
  \mathrm{\lor}(x, y) = N( N(x) \land N(y) ),
\]

\noindent therefore $\mathrm{eq}(x, y) = N(\mathrm{gt}(x, y)) \land N(\mathrm{lt}(x, y))$.

On the other hand,  we construct $\Sigma$ operator in the following way:

\[
 \begin{array}{c}
    \displaystyle \mathrm{pow}(x, k)  = \prod_{i = 1}^{k} P_2^2(i, x), \\
    \displaystyle \log (x) = \prod_{i = 0}^{x} i^{N(\mathrm{eq}(2^i, x))}, \\
    \displaystyle \sum_{i=n}^{m} f(i, \mathbf{x}) = \log \prod_{i=n}^{m} 2^{f(i, \mathbf{x})}, \\
 \end{array}
\]

\noindent and the rest of our proof is trivial:
  $\displaystyle x\times y = \sum_{i = 1}^x y$, 
  $\displaystyle x + y = \log \left( 2^{x} \times 2^y \right)$, 
  $\displaystyle x - y = \sum_{i = 0}^x N(\eq(i + y, x)) \times i$ and 
  $\displaystyle |x-y| = \mathrm{gt}(x, y) \times (x - y) + \mathrm{lt}(x, y) \times (y - x)$. \qed

\end{pf}


\subsection{Let $M(x)$ be $M(M(x+11))$ when $x\leq 100$ and $x-10$ when $x>100$.
Prove $M(x)=91$ when $x\leq 100$.}
\begin{pf} \rm
The basic case is $M(99) = M(M(110)) = M(100) = M(M(111)) = M(101) = 91$, and
$M(x) = M(M(x)) = M(x + 1)$ when $90 \leq x \leq 100$.
An induction on $x$ shows $M(x)=91$ for all $0\leq x\leq 100$. \qed
\end{pf}

\subsection{Prove: $\min x\leq n.[f(x,\mathbf{y})] = n - \max x \leq n.[f(n - x,\mathbf{y})]$,
and $\max x\leq n.[f(x,\mathbf{y})] = n - \min x \leq n.[f(n - x,\mathbf{y})]$.}

\begin{pf} \rm
 For simplicity, let $m=\min x\leq n.[f(x,\mathbf{y})]$ and $M=\max x\leq n.[f(n - x,\mathbf{y})]$.

If there is no $0\leq x\leq n$ satisfying $f(x,\mathbf{y})=0$, we have $m=n$ and $M=0$, hence
$m + M = n$. Otherwise, let $a$ be the minimum root of $f(x, \mathbf{y})$, thus 
$f(x, \mathbf{y})\neq 0$ for all $x<a$, and  $f(n - x, \mathbf{y})\neq 0$ for all $x > n - a$.
By definition, we can easily see that $m+M=n$. Since both $m$ and $M$ will not exceed $n$, 
$m+M=n$ yields $m = n - M$ and $M = n - m$. 

The another case is trivial by symmetry. \qed
\end{pf}

\subsection{Prove: $\mathcal{EF}$ is closed under the bounded $\max$ operator.}

\begin{pf} \rm For any $f \in \EF$,

\noindent
$\displaystyle \max x \leq n.[f(x,\mathbf{y})]
= \sum_{i=0}^{n} \left[ \left\lfloor
  \left( \sum_{x=0}^{i} N(x, \mathbf{y}) \right)
   /
  \left( \sum_{x=0}^{n} N(x, \mathbf{y}) \right) \right\rfloor
  \times i
\right].
$ \qed
\end{pf}

\subsection{Prove: Euler's totient function $\varphi\in\EF$.}
\begin{pf} \rm
 $\displaystyle \varphi(x) = 
\left\{ \sum_{y=0}^{n} N\left[
\left( \sum_{d=0}^{x+y} \Big| \mathrm{rs}(x, d) - \mathrm{rs}(y, d)\Big| \right) - 2  \right]
\right\}
- 1$ .\qed
\end{pf}

\subsection{Let $h(x)$ be subscript of the greatest prime factor. Assume that $h(0) = h(1)=0$,
prove that $h\in\EF$.}
\begin{pf} \rm
$\displaystyle h(x) = \max i\leq x. \left\{
N^2\left| \sum_{j = 0}^{i} [N(\mathrm{rs}(i, j))] - 2 \right|
+ N^2[\mathrm{rs}(x, i)] \right\}
$.\qed
\end{pf}

\subsection{Prove that the Fibonacci sequence $f(0)=f(1)=1,f(x+2)=f(x)+f(x+1) \in \EF$ and $\PRF$.}
\begin{pf} \rm
  Let $\{\pg, K, L\}$ be any paring function in $\PRF$. Let
\[
\begin{array}{c}
F(0) = \pg( 1, 0 ) \\
F(x + 1) = \pg(K(F(x)) + L(f(x)),  K(F(x))),
\end{array}
\]
we have $F$ is in $\PRF$ and $K(F(x)) = f(x)$, therefore $f\in\PRF$. 

On the other hand, 
$f(x)$ is the number of binary strings of length $x-1$ without successive 1s. Therefore


$\displaystyle f(x) = \sum_{i=0}^{2^{n-1}} \sum_{j = 0}^{n - 2}
N\Big(\eq ( \rs(i, 2^j) , \rs(i, 2^{j + 1}))\Big) \times
\eq \Big( \rs(i, 2^j) , 0 \Big) \in \EF$.
\qed
\end{pf}

\subsection{Prove that the number theoretic function $Q(x,y,z,v)\equiv p(
\langle x, y, z\rangle)~|~v$ is elementary.}

\begin{pf} \rm
 We have already seen that $p(n)\in\EF$ and
 $\langle x, y, z\rangle = 2^x\cdot 3^y\cdot 5^z \in\EF$.
 Therefore $Q(x, y, z) = \eq(\rs(v, p(\langle x, y, z\rangle)), 0) \in\EF$. \qed
\end{pf}

\subsection{Let $f:\mathbb{N}\to\mathbb{N}$, $f(0) = 1, f(1) = 4, f(2) = 6,
  f(x+3)=f(x)+f^2(x+1)+f^3(x+2)$. Prove that $f\in\PRF$.}

\begin{pf} \rm
  Let $G(0) = \langle 1, 4, 6\rangle$ and
\[
 G(x + 1) = \langle
   \ep_1(G(x)),
   \ep_2(G(x)),
   \ep_0( G(x) ) + \ep_1^2(G(x)) + \ep_2^3(G(x))
 \rangle,
\]

\noindent we have $\mathrm{ep}_0(G(x)) = f(x)$. \qed
\end{pf}

\subsection{Let $f(n) = n^{n^{\ldots^n}}$, prove that $f\in \PRF-\EF$.}

\begin{pf} \rm
 Let $g(n, 0) = 0$ and 
 $g(n, x + 1) = n^{ g(n, x)}$. Thus $g\in \PRF$ and $g(n, n) = f(n)$,
 therefore $f\in\PRF$. On the other hand, $G(k, x) = 2^{2^{\ldots^x}}$ is
 one among the control functions of $\EF$. If $f\in EF$, there exists
 $k$ such that $G(k, n) > f(n)$ for all $n$. However, this is impossible
 because $f(k + 2)$ is always greater than $G(k, k)$. \qed
\end{pf}


\subsection{Let $g:\mathbb{N}\to\mathbb{N}\in\PRF,
f:\mathbb{N}^2\to\mathbb{N}$ satisfies that
$f(x,0) = g(x)$, $f(x, y + 1) = f(f(\ldots f(f(x, y), y - 1), \ldots),0)$.
Prove that $f\in\PRF$.}

\begin{pf} \rm
 Let $G(x, 0) = x$ and $G(x, y + 1) = g(G(x, y))$,
 $F(0) = 1, F(x + 1) = F(x) + \sum_{i = 0}^{x} F(x)$. it is obvious that
 $G\in\PRF$, and $F(x) = Fib(2x) \in \PRF$.
 
 We now prove that $f(x, y) = G(x, F(y))$.
 The basis is $f(x, 0) = G(x, 1) = g(x)$, and we assume that
 $f(x, y^*) = G(x, F(y^*))$ For all $y^* \leq y$. Therefore, 

\[
 \begin{array}{rcl}
 f(x, y + 1)  & = &  \underbrace{g(g(\ldots g(}_{\sum_{i = 0}^{y} F(i)\textrm{ times}}
   f(x, y) )\ldots )) \\
 & = & \underbrace{g(g(\ldots g(}_{F(y) + \sum_{i = 0}^{y} F(i)\textrm{ times}}
   x )\ldots )) \\
   & = & G(x, F(y + 1)),
  \end{array}
\]

\noindent which means $f(x, y) = G(x, F(y))\in \PRF$. \qed
\end{pf}


\subsection{If $f,g:\mathbb{N}\to\mathbb{N}$ differs for only finitely many
values. Prove that $f\in\RF$ if and only if $g\in\RF$. }


\subsection{Prove that $
\displaystyle\left\lfloor\left(\frac{\sqrt{5}+1}{2}\right)n\right\rfloor \in \EF$.}

\begin{pf} \rm
Let $\varphi = \frac{\sqrt{5}+1}{2}$, we can rewrite the solution of 
$y=\lfloor \varphi n\rfloor$ by

\[
\begin{array}{rcl}
 y &= & \displaystyle \max_{x\in\mathbb{N}} x \\
 & \mathrm{s.t.} & \varphi n \leq x,
\end{array}
\]
\noindent
therefore
$ \displaystyle y = 
 \sum_{i = 0}^{2n}
 i\times N \left\{
 \eq \left[
  \sum_{j = i}^{2n} N\Big(\eq(i^2 - in - n^2, 0)\Big)
 , 1
 \right] \right\} $.
\qed
\end{pf}

\subsection{1.20}

\subsection{1.21}

\subsection{1.22}

\subsection{1.23}

\subsection{Define $g:\mathbb{N}\to\mathbb{N}$ by 
$g(0) = 0, g(1) = 1, g(n + 2) = \rs((2002g(n + 1) + 2003g(n)), 2005)$. Find $g(2006)$.}

\begin{pf} \rm
We have $\displaystyle g(n) = \rs \left( \frac{(-1)^{n+1} + 2003^n}{2004} , 2005 \right)$
and $2005=5\cdot 401$, therefore
 
\[
 \begin{array}{rcl}
   g(2006) \mod 2005 & = & \Big((2003^{2006} - 1) \times 2004^{-1} \Big) \mod 2005 \\
                     & = & \Big((2^{2006} - 1) \times 2004\Big) \mod 2005. \\
 \end{array}
\]

Since $a^{p-1} \equiv 1\mod p$ for all prime $p$,
$2^{2006} \equiv 2^2 \equiv 4\mod 5, 2^{2006} \equiv 2^6 \equiv 64 \mod 401$.
According to the Chinese remainder theorem,
$2^{2006} \equiv 64\mod 2005$. Therefore, $g(2006) \equiv 63\times2004
\equiv 1942\mod 2005$. \qed

\end{pf}

\subsection{1.25}


\newpage
\end{document}