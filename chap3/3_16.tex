\problem{Prove that for all $M,N\in\Lambda$, $M\equivbeta N$ implies the existence
of $T$ such that $M\manystepbeta T$ and $N\manystepbeta T$.}
\begin{pf} \rm
  $M\equivbeta N$ implies that 
\[
(M,N)\in
\bigcup_{k=0}^{\infty} (\onestepbeta \cup \leftarrow_\beta)^k.
\]
We take the basis case of $k=0$, that is, $M\equiv N$ as our basis.
Assume that for all $(M,N)\in (\onestepbeta \cup \leftarrow_\beta)^k$, there exists
$T\in\Lambda$ such that $M\manystepbeta T$ and $N\manystepbeta T$. For the case of
$(M,N)\in (\onestepbeta \cup \leftarrow_\beta)^{k+1}$, either
\[
 (1)~ M\onestepbeta P \equivbeta N
\]
or
\[
 (2)~ M\leftarrow_\beta P\equivbeta N
\]
holds where $(P,N)\in (\onestepbeta \cup \leftarrow_\beta)^k$.
According to the inductive hypothesis, there exists $T_0$ such that
$P\manystepbeta T_0$ and $N\manystepbeta T_0$. Because $\manystepbeta$
is transitive, We have $M\manystepbeta T_0$ in the case $(1)$.

In the case $(2)$, according to the CR-property of $\manystepbeta$,
$P\manystepbeta M$ and $P\manystepbeta T_0$ implies the existence of
$T\in\Lambda$ that, $M\manystepbeta T$ and $T_0\manystepbeta T$.
Again because the transitivity of $\manystepbeta$, $N\manystepbeta T_0$ and
$T_0\manystepbeta T$ yields $N\manystepbeta T$.
Therefore such a $T$ exists for all $k\in\mathbb{N}$.
 \qed
\end{pf}
